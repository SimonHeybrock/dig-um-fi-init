\documentclass{article}

\usepackage{url}
\usepackage{graphicx}
\usepackage{xcolor}
\usepackage[utf8]{inputenc}
\usepackage{natbib}
\bibliographystyle{agsm}
\usepackage{mathpazo}

\definecolor{ivoacolor}{rgb}{0.0,0.318,0.612}
\definecolor{linkcolor}{rgb}{0.318,0,0.318}

\RequirePackage[colorlinks,
	linkcolor=linkcolor,
	anchorcolor=linkcolor,
	citecolor=linkcolor,
	urlcolor=linkcolor,
	breaklinks=true]{hyperref}

\newcommand{\sectauthor}[1]{\marginpar{\footnotesize by #1}}


\title{Federated Infrastructures in Research on Universe and Matter:
State of Play}
\author{DIG-UM Topic Group Federated Infrastructures}

\begin{document}
\maketitle
\begin{abstract}
As the DIG-UM Topic Group on Federated Infrastructures begins its work,
this document tries to provide a concise and necessarily subjective
overview over the state of play of digital research infrastructures in
the domains covered by the eight committees.  Its main goal is to
help the comittee members understand the practices and technologies
already established in the other domains.  It may also be useful to
identify progress made as DIG-UM progresses.

\end{abstract}

\section{Introduction}

The following sections were written by topic group members nominated for
the communities represented by the various committees.  They in
particular try to answer the following questions:

\begin{itemize}
\item Which federated infrastructures are already operated in the
respective communites?
\item Which technologies are being used in federation and service
provision?
\item What are the experiences with these infrastructures?  Are there
lessons learned?
\item Which further infrastructures should be made interoperable and/or
federated?  What should be enabled by the federation?
\end{itemize}

\input rds
\end{document}
